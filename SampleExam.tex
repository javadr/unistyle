\documentclass[a4paper,12pt]{unistyle}

\usepackage{url}
\usepackage{amsmath}

\unidetails{
عنوان درس=مبانی کامپیوتر و برنامه‌سازی,
مقطع و رشته=ک‍‍ارش‍ن‍‍اس‍‍ى‌ ‌ع‍ل‍وم‌‌ک‍‍ام‍پ‍ی‍وت‍ر,
روز و تاریخ=دوشنبه \LR{1398/11/01},
ترم=1,
سال تحصیلی=98--99,
مدت امتحان=180,
نوع آزمون=پایان‌ترم,
لوگوی دانشگاه=logo,
}


\begin{document}
\MakeTitle
\begin{enumerate}

    \item
    پرسش‌های زیر را در \textbf{همین برگه} پاسخی کوتاه دهید. \grade{2}
    \begin{enumerate}[itemtwocol]
        \parskip=20pt
        \item تنها با یک دستور و  در یک خط سه متغیر تعریف نموده و مقدار اولیه آنها را صفر قرار دهید.
        \item \LRE{\texttt{"\{0\} * \{1\} is not \{0\}".format("*", 5)}}؟
    \end{enumerate}


    \item سوال تستی چهارستونی ستونی       \grade{4}
        \fourj{گزینه اول}{گزینه دوم}{گزینه سوم}{گزینه چهارم}
    \item سوال تستی دوستونی ستونی
        \twoj{گزینه اول}{گزینه دوم}{گزینه سوم}{گزینه چهارم}
    \item سوال تستی تک‌ستونی ستونی
        \onej{گزینه اول}{گزینه دوم}{گزینه سوم}{گزینه چهارم}

\end{enumerate}

\makeresponseform{40}

\sign
\kalamehakim
\vfill\xepersianproof

\end{document}
