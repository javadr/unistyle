\documentclass[a4paper,12pt,flowchart,key]{unistyle}

\unidetails{
عنوان درس=مبانی کامپیوتر و برنامه‌سازی,
مقطع و رشته=ک‍‍ارش‍ن‍‍اس‍‍ى‌ ‌ع‍ل‍وم‌‌ک‍‍ام‍پ‍ی‍وت‍ر,
روز و تاریخ=دوشنبه \LR{1398/11/01},
ترم=1,
سال تحصیلی=98--99,
مدت امتحان=180,
نوع آزمون=پایان‌ترم,
لوگوی دانشگاه=logo,
}

\begin{document}
\MakeTitle

\begin{enumerate}
    \item \null
    \begin{multicols}{2}
    کامپیوتری را تصوّر نمایید که در آن تنها جمع با عدد یک و کسر یک واحد تعریف شده باشد؛ بعبارت دقیق‌تر در این ماشین تنها اعمال
    $x+1$  و $x-1$  تعریف شده‌اند و عبارت $x+y$  زمانیکه هر دو شناسه عددی غیر از یک باشد منجر به خطا خواهد شد.
    می‌خواهیم در این ماشین دو تابع بازگشتی برای اعمال جمع و تفریق دو عدد صحیح بنویسیم. \grade{2}
    \vfill
\begin{lstlisting}
def add(m,n):
	if m<0 and n>=0: m,n=n,m
	if m==0: return n
	if n==0: return m
	return add(m-1,n+1)
\end{lstlisting}
    \end{multicols}
    شرط اول اگر نباشد تابع برای حالتی که عدد اول منفی و دومی مثبت باشد در حلقه بی‌پایان خواهد افتاد. ضمناً تابع تفریق
    هم به طریقی مشابه نوشته می‌شود؛ چگونه؟

   \item
       فلوچارتی را رسم کنید که با گرفتن یک عدد صحیح از ورودی مشخص نماید که آن عدد، عددی اول است یا خیر؟ \grade{2}

\begin{center}
\scalebox{.9}{
\begin{tikzpicture}[node distance=2cm,scale=.5]
    \node (start) [startstop] {شروع};
    \node (in) [io, right of=start, xshift=1cm] {n};
    \node (pro1) [process, right of=in, xshift=1cm] {$i\leftarrow 2$};
    \node (dec1) [decision, right of=pro1, xshift=1cm] {$i<n$};
    \node (dec2) [decision, right of=dec1, xshift=2cm] {$n\%i=0$};
    \node (pro2) [process, below of=dec2, yshift=-.1cm, xshift=-2cm] {$i\leftarrow i+1$};
    \node (out1) [io, right of=dec2, xshift=2cm] {\rl{ عدد مرکب}};
    \node (out2) [io, above of=dec2, ] {\rl{ عدد اول}};
    \node (end) [startstop, above of=out1, ] {پایان};

    \draw [arrow] (start) -- (in);
    \draw [arrow] (in) -- (pro1);
    \draw [arrow] (pro1) -- (dec1);
    \draw [arrow] (dec1) -- node [anchor=south] {بله} (dec2);
    \draw [arrow] (dec2) |- node [anchor=west, near start] {خیر} (pro2);
    \draw [arrow] (pro2) -| (dec1);
    \draw [arrow] (dec2) -- node [anchor=south] {بله} (out1);
    \draw [arrow] (out1) -- (end);
    \draw [arrow] (out2) -- (end);
    \draw [arrow] (dec1) |- node [anchor=east, near start] {خیر} (out2);
\end{tikzpicture}
}
\end{center}

   \item
   \begin{enumerate}
       \item خطاهای نگارشی برنامه زیر را تصحیح نمایید --در همین برگه مشخص گردد--. \grade{2}
       \item مشخص نمایید که خروجی این برنامه چه چیزی خواهد بود.  ---  $x^k$  --- \grade{1}
   \end{enumerate}

    \begin{latin}
    \begin{lstlisting}
    x = Input("Please enter a number: ") # int(input...
    k = input '(Please enter another number: )'   #int ... ('')
    y = x
    a = 1
    i = K # k
    While i > 0 # while i>0:
        if  i % 2 == 1:
            a = a * y
        y = y * y
        i = i//2
    print "the output is {}".format(a) # print ()
    \end{lstlisting}
    \end{latin}

\end{enumerate}
\sign
\null\vfill

\null\vfill\kalamehakim[1]\xepersianproof
\end{document}
